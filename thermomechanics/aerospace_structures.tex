\chapter{Aerospace Structures}

\newpage
\section{Airworthiness}



\subsection{Regulations}
\hl{some very condensed regulation can go here}





\newpayge
\section{Airframe Loads}
\subsection{Aerodynamic Loads}

\subsection{Inertial Loads}

\subsection{Flight Maneouvers}
\subsubsection{Level Flight}

\subsubsection{Steady Pull-Out}

\subsubsection{Correctly Banked Turn}

\subsubsection{Landing}

\subsubsection{Gust Loads}


\subsection{V-n Diagrams}

\subsection{Wing Loads on Flight Phases}


















\newpage
\section{Aeroelasticity}
Aeroelasticity is essentially the coupling of aerodynamics loads and the structural elastic behavior of the airplane's fuselage, and specially, tail and wings. What happens in the wing for example is the lift generated bends the wing upward changing the local incidence angle thus adding more lift, thus more bending, thus more lift, till a equilibrium is reached, if the wing stiffness if enough or till the material load limit and therefore destruction of the wing.

Aeroelasticity can subdivided into steady and dynamic phenomena. In steady
aeroeslaticity, we assume that a equilibrium of aerodynamic loading distribution and the deformed structured has been reached thus assuming that that configuration exists by providing enough stiffness.  In dynamic aeroelasticity were interest in analyzing the transient behavior of the structure, thus inertial forces must be included in the models.

\subsection{Airfoil Torsional Divergence}
Coupling of torsional load and the divergence is one of the most important in wing aeroelasticity because a twist of the wing greatly impacts local incidence angle of the airfoil and thus the lift and looping back the the structure twist.
Simple 2-D wing tunnel models are presented next:

\subsubsection{Wall Mounted Model}

\subsubsection{Sting Mounted Model}

\subsubsection{Strut Mounted Model}

\subsection{Finite Straight Wing Torsional Divergence}
Making an equilibrium of moments in spanwise wing element of lenght $\Delta z$ one obtains the following equation
\begin{align}
\left( T+\frac{d T}{d z}\Delta z \right)-T +ec\Delta L + \Delta M_{ac}=0\\
\Delta L= \frac{1}{2} \rho U^2 \Delta S C_L = \frac{1}{2} \rho U^2 c \Delta z \left( C_{L_0} + \frac{\partial C_L}{\partial \alpha}(\alpha + \theta) \right)\\
\Delta M_{ac}= \frac{1}{2} \rho U^2 \Delta S c Cm_{ac} = \frac{1}{2} \rho U^2 c^2 \Delta z Cm_{ac}\\
T=GJ \frac{d \theta}{dz}
\end{align}
Combining the above equations one obtains
\begin{align}
GJ\frac{d^2 \theta}{dz^2} + \frac{1}{2} \rho U^2 ec^2 \left( C_{L_0} + \frac{\partial C_L}{\partial \alpha}(\alpha + \theta) \right) + \frac{1}{2} \rho U^2 c^2 Cm_{ac}=0 \label{ode_theta}
\end{align}
Equation \eqref{ode_theta}is actually an second order differential equation in $\theta(z)$, and it admits a solution of following type.
\begin{align}
\theta= A sin(\lambda z) + B cos(\lambda z) - \left(\frac{C_{m_0}}{e C_{l_\alpha}} + \alpha \right)
\end{align}




\subsection{Finite Swept Wing Torsional Divergence}
The swept back wing decreases the the
