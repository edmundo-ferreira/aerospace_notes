\chapter{Composite Materials}

\section{Micro Mechanics}
Micro Mechanics deals only the determination of the characteristics of the \emph{ply} from the its basic constituents, matrix and fiber.
One very common parameter of the ply material is the fractional volumes of each component on a composite ply of volume $V_c$.
\begin{equation}
v_i=\frac{V_i}{V_c}=\frac{V_i}{\sum\limits_{i} V_i}
\end{equation}

The determination of the height of each ply $h_i$, could be determined from the grammage ($\sfrac{kg}{m^2}$) of the fiber in the composite ply $m_{\square f}$.

\begin{gather}
\begin{align}
\rho_f &= \frac{m_f}{V_f}=\frac{m_f}{v_f V_c} \nonumber \\
\rho_f &= \frac{m_f}{v_f 1 hi} \qquad \quad  \text{for a volume of $1m^2$ and height $h_i$}\nonumber\\[1mm]
h_i &= \frac{m_{\square f}}{v_f \rho_f}
\end{align}
\end{gather}
The total height of \emph{laminate} $h_t$, made up from $n$ layers of ply is:

\begin{gather}
\begin{align}
h_t&=\sum\limits_{i}^{n}h_i\\
h_t&=nh_i \quad \text{if $h_i$ is equal on each layer}
\end{align}
\end{gather}

We start off by a mass of the composite ply $m_c$, from the masses of fiber $m_f$, and mass of matrix $m_m$.
\begin{subequations}
\begin{align}
m_{c}&=m_{m}+m_{f} \\
\rho_c V_c&= \rho_m V_m + \rho_f V_f \nonumber \\
\rho_c & = \rho_m v_m + \rho_f v_f \label{mixture_law}
\end{align}
\end{subequations}

Equation \eqref{mixture_law} is know as the mixture law and plays a pivot role in the determination of other mechanical properties of each ply.

\begin{gather}
\begin{bmatrix}
\epsilon_{xx}\\
\epsilon_{yy}\\
\epsilon_{zz}\\
\epsilon_{yz}\\
\epsilon_{xz}\\
\epsilon_{xy}\\
\end{bmatrix}=
\begin{bmatrix}
\frac{1}{E_x} & -\frac{\nu_{xy}}{E_y} &  -\frac{\nu_{xz}}{E_z} & 0& 0&0\\[1mm]
-\frac{\nu_{yx}}{E_x} & \frac{1}{E_y} &  -\frac{\nu_{yz}}{E_z} & 0& 0&0\\[1mm]
-\frac{\nu_{zx}}{E_x} & -\frac{\nu_{zy}}{E_y} &  \frac{1}{E_z} & 0& 0&0\\[1mm]
0 & 0 & 0 & \frac{1}{G_{yz}}  & 0 &0 \\[1mm]
0 & 0 & 0 & 0  & \frac{1}{G_{xz}} &0 \\[1mm]
0 & 0 & 0 & 0  & 0 &\frac{1}{G_{xy}} \\[1mm]
\end{bmatrix}=
\begin{bmatrix}
\sigma_{xx}\\
\sigma_{yy}\\
\sigma_{zz}\\
\sigma_{yz}\\
\sigma_{xz}\\
\sigma_{xy}\\
\end{bmatrix} \label{ortho_system}\\[2mm]
\frac{\nu_{ij}}{E_{i}}=\frac{\nu_{ji}}{E_{j}}
\end{gather}


For plane stress (around the z axis), the system \eqref{ortho_system} can be reduced since $\sigma_{zz}=\sigma_{xz}=\sigma_{yz}=0$.



\begin{subequations}
\label{ply_system}
\begin{align}
\begin{bmatrix}
\epsilon_{xx}\\
\epsilon_{yy}\\
\epsilon_{xy}\\
\end{bmatrix}&=
\begin{bmatrix}
\frac{1}{E_x} & -\frac{\nu_{xy}}{E_y} &  0\\[1mm]
-\frac{\nu_{yx}}{E_x} & \frac{1}{E_y} &  0\\[1mm]
0 & 0 &\frac{1}{G_{xy}}\\[1mm]
\end{bmatrix}=
\begin{bmatrix}
\sigma_{xx}\\
\sigma_{yy}\\
\sigma_{xy} 
\end{bmatrix} \label{ply_matrix} \\[2mm] 
\epsilon_{zz}&=- \left(\frac{\nu_{zx}}{E_x}\sigma_{xx} +\frac{\nu_{zy}}{E_y}\sigma_{yy}\right)
\end{align}
\end{subequations}
\vspace{-6mm}
\begin{align}
\frac{\nu_{xy}}{E_{x}}=\frac{\nu_{yx}}{E_{y}} \label{ply_sym}
\end{align}

Equations \eqref{ply_system} and \eqref{ply_sym} provide the fundamental elastic relations of the ply system. It is clear from \ref{ply_matrix} that the constitutive relations only depend of four mechanical properties, $E_x$, $E_y$, $\nu_{xy}$ and $G_{xy}$.
\newpage
\textbf{Structural Teories} (Reddy Chapter 3)

\begin{enumerate}
\item[a] Reduced dimension theories (2-D and 1-D)
	\begin{enumerate}
	\item Classical (no shear deformation)
	\item 1\textsuperscript{st} order shear deformation
	\item 2\textsuperscript{nd} order shear deformation
	\item 3\textsuperscript{rd} order shear deformation
	\end{enumerate}
\item[b] Three-dimensional elasticity theory (3-D)
	\begin{enumerate}
	\item Traditional 3D elasticity formuations
	\item Layerwise theories
	\end{enumerate}
\item[c] Multiple model methods (2-D and 3-D)
	
\end{enumerate}

 
