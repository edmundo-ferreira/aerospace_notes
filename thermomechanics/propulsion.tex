\chapter{Propulsion Systems}
    
    The fuel to inlet flow ratio is defined as follows:
    \begin{equation}
        f \defeq \frac{\dot{m}_f}{\dot{m}_i}
    \end{equation}
    
%    Some usefull performance measurements are the propulsive power efficiency $\eta_{propulsive}$, thermal efficiency $\eta_{thermal}$  
    %\begin{align}
        %\label{propulsive_efficiency}
        %\eta_{propulsive} &\defeq& \frac{\text{Thrust Power}}{\text{Jet Power}} = \frac{T u}{P_{jet}} \\
        %\label{thermal_efficiency}
        %\eta_{thermal} &\defeq& \frac{\text{Jet Power}}{\text{Fuel Calorific Power}}   = \frac{P_{jet}}{\dot{Q}_R\dot{m}_f}\\
        %\label{total_efficiency}
        %\eta_{total} &\defeq& \frac{\text{Thrust Power}}{\text{Fuel Calorific Power}} = \eta_{propulsive} \eta_{thermal}
    %\end{align}
 
    A force parameter that is usually very importat is the specific impulse $I_s$ since it usually defines the size of the engine:
    \begin{equation}
        I_s \defeq \frac{|\vec{T}|}{\dot{m}_f} 
    \end{equation}
    
    The consumption for fuel to thrust is also a very important parameter and it is named specific fuel consumption, and can be calculated for jet or prop engines with slightly different names 
    \begin{align}
       TSFC  &\defeq& \frac{\dot{m}_f}{|\vec{T}|} \\
       BSFC  &\defeq& \frac{\dot{m}_f}{\dot{W}_u} \\
       EBSFC &\defeq& \frac{\dot{m}_f}{\dot{W}_u + |\vec{T}_e|v_e} \\
    \end{align}
 


\newpage
\section{Ram/ScramJet Engine}
    \subsection{Thrust Force}
    \begin{equation}
        T_x = \dot{m}_e v_e - \dot{m}_i v_i
    \end{equation}
 
    \subsection{Ideal Thermodynamic Cycle}
    
    
    

\newpage
\section{Turbojet Engine}
    \subsection{Thrust Force}
    Mass Balance:
    \begin{align}
        \dot{m}_e &= &\dot{m}_i + \dot{m}_f\\
        \dot{m}_e &= &\dot{m}_i ( 1 + \frac{\dot{m}_f}{\dot{m}_f}) = \dot{m}_i ( 1 + f) 
    \end{align}

    Momentum Balance:
    \begin{align}
        T_x &=& \dot{m}_e v_e - \dot{m}_i v_i + A_e (p_e - p_\infty)\\
        T_x &=& \dot{m}_i[(1+f)v_e - v_i] + A_e (p_e - p_\infty)\\
    \end{align}
    If we assume $p_e = p_\infty$:
    \begin{align}
        T_x &=& \dot{m}_i[(1+f)v_e - v_i] 
    \end{align}
%    \begin{align}
        %\eta_{propulsive} = \frac{T v}{P_{jet}} \\
    %\end{align}
    %\begin{align}
        %P_{jet} =  \dot{m}_i (v_i)^2\\
    %\end{align}
    

    \subsection{Ideal Thermodynamic Cycle}



\newpage
\section{TurboFan Engine}

\newpage
\section{Rocket Engine}
    \subsection{Trust Force}
    Similar to the other propulsive devices the rocket can be thought as a particular case where the propulsion device has no inlet mass flow $\dot{m}_i = 0$.
    \begin{equation}
        \label{rocket_thrust}
        T_x = \dot{m}_e v_e + A_e (p_e - p_\infty)
    \end{equation}
    \subsection{Ideal Thermodynamic Cycle}
    The rocket propulsion cycle will be also very simplified by the fact that there is no inlet mass flow therefore it only contains a combustion chamber (the interior of the rocked itself in case its solid fuel) and a converging-diverging nozzle in order to increase the speed of the combustion products till sonic caracteristics at the throat and supersonic at the noze outlet, avoid in all operating ranges normal chock waves inside the nozzle since it could cause damage to the engine and also a huge penalty in the engine performance. 

    (put T-S diagram for this cycle)         
    

    \subsection{Vehicle Dynamics}
    The vehicle dyamics is quite interesting due to the fact the coupling with the propulsive force is great.
    The momentum balance to the entire system can be tought as a compound control volume that is acelerated himself therefore the Reynolds trasport equation is:  
    \begin{align}
        \frac{d \vec{p}}{dt}|_{\text{sys}} &=& \vec{F}_{\text{ext\_sys}}\\ 
        \frac{d \vec{p}}{dt}|_{\text{rel\_cv}} + \frac{d \vec{p}}{dt}|_{\text{cv}} &=& \vec{F}_{\text{ext\_sys}}\\ 
        \int_V \rho \, \vec{a}_{cv} dV + \int_V \frac{\partial (\rho v_i)}{\partial t} dV + \int_A \rho v_i v_j n_j dA&=& \vec{F}_{\text{ext\_sys}} \\
        \int_V \rho \frac{\vec{V}_{cv}}{dt} dV + 0 + \dot{m}_e v_e &=& m\vec{g}
    \end{align} 
    \begin{equation}
        m\frac{\vec{V}_R}{dt} = -\dot{m} v_e  + m\vec{g}
    \end{equation}


